% Options for packages loaded elsewhere
\PassOptionsToPackage{unicode}{hyperref}
\PassOptionsToPackage{hyphens}{url}
%
\documentclass[
]{article}
\title{Homework One}
\author{Rachel Meredith}
\date{1/26/2022}

\usepackage{amsmath,amssymb}
\usepackage{lmodern}
\usepackage{iftex}
\ifPDFTeX
  \usepackage[T1]{fontenc}
  \usepackage[utf8]{inputenc}
  \usepackage{textcomp} % provide euro and other symbols
\else % if luatex or xetex
  \usepackage{unicode-math}
  \defaultfontfeatures{Scale=MatchLowercase}
  \defaultfontfeatures[\rmfamily]{Ligatures=TeX,Scale=1}
\fi
% Use upquote if available, for straight quotes in verbatim environments
\IfFileExists{upquote.sty}{\usepackage{upquote}}{}
\IfFileExists{microtype.sty}{% use microtype if available
  \usepackage[]{microtype}
  \UseMicrotypeSet[protrusion]{basicmath} % disable protrusion for tt fonts
}{}
\makeatletter
\@ifundefined{KOMAClassName}{% if non-KOMA class
  \IfFileExists{parskip.sty}{%
    \usepackage{parskip}
  }{% else
    \setlength{\parindent}{0pt}
    \setlength{\parskip}{6pt plus 2pt minus 1pt}}
}{% if KOMA class
  \KOMAoptions{parskip=half}}
\makeatother
\usepackage{xcolor}
\IfFileExists{xurl.sty}{\usepackage{xurl}}{} % add URL line breaks if available
\IfFileExists{bookmark.sty}{\usepackage{bookmark}}{\usepackage{hyperref}}
\hypersetup{
  pdftitle={Homework One},
  pdfauthor={Rachel Meredith},
  hidelinks,
  pdfcreator={LaTeX via pandoc}}
\urlstyle{same} % disable monospaced font for URLs
\usepackage[margin=1in]{geometry}
\usepackage{color}
\usepackage{fancyvrb}
\newcommand{\VerbBar}{|}
\newcommand{\VERB}{\Verb[commandchars=\\\{\}]}
\DefineVerbatimEnvironment{Highlighting}{Verbatim}{commandchars=\\\{\}}
% Add ',fontsize=\small' for more characters per line
\usepackage{framed}
\definecolor{shadecolor}{RGB}{248,248,248}
\newenvironment{Shaded}{\begin{snugshade}}{\end{snugshade}}
\newcommand{\AlertTok}[1]{\textcolor[rgb]{0.94,0.16,0.16}{#1}}
\newcommand{\AnnotationTok}[1]{\textcolor[rgb]{0.56,0.35,0.01}{\textbf{\textit{#1}}}}
\newcommand{\AttributeTok}[1]{\textcolor[rgb]{0.77,0.63,0.00}{#1}}
\newcommand{\BaseNTok}[1]{\textcolor[rgb]{0.00,0.00,0.81}{#1}}
\newcommand{\BuiltInTok}[1]{#1}
\newcommand{\CharTok}[1]{\textcolor[rgb]{0.31,0.60,0.02}{#1}}
\newcommand{\CommentTok}[1]{\textcolor[rgb]{0.56,0.35,0.01}{\textit{#1}}}
\newcommand{\CommentVarTok}[1]{\textcolor[rgb]{0.56,0.35,0.01}{\textbf{\textit{#1}}}}
\newcommand{\ConstantTok}[1]{\textcolor[rgb]{0.00,0.00,0.00}{#1}}
\newcommand{\ControlFlowTok}[1]{\textcolor[rgb]{0.13,0.29,0.53}{\textbf{#1}}}
\newcommand{\DataTypeTok}[1]{\textcolor[rgb]{0.13,0.29,0.53}{#1}}
\newcommand{\DecValTok}[1]{\textcolor[rgb]{0.00,0.00,0.81}{#1}}
\newcommand{\DocumentationTok}[1]{\textcolor[rgb]{0.56,0.35,0.01}{\textbf{\textit{#1}}}}
\newcommand{\ErrorTok}[1]{\textcolor[rgb]{0.64,0.00,0.00}{\textbf{#1}}}
\newcommand{\ExtensionTok}[1]{#1}
\newcommand{\FloatTok}[1]{\textcolor[rgb]{0.00,0.00,0.81}{#1}}
\newcommand{\FunctionTok}[1]{\textcolor[rgb]{0.00,0.00,0.00}{#1}}
\newcommand{\ImportTok}[1]{#1}
\newcommand{\InformationTok}[1]{\textcolor[rgb]{0.56,0.35,0.01}{\textbf{\textit{#1}}}}
\newcommand{\KeywordTok}[1]{\textcolor[rgb]{0.13,0.29,0.53}{\textbf{#1}}}
\newcommand{\NormalTok}[1]{#1}
\newcommand{\OperatorTok}[1]{\textcolor[rgb]{0.81,0.36,0.00}{\textbf{#1}}}
\newcommand{\OtherTok}[1]{\textcolor[rgb]{0.56,0.35,0.01}{#1}}
\newcommand{\PreprocessorTok}[1]{\textcolor[rgb]{0.56,0.35,0.01}{\textit{#1}}}
\newcommand{\RegionMarkerTok}[1]{#1}
\newcommand{\SpecialCharTok}[1]{\textcolor[rgb]{0.00,0.00,0.00}{#1}}
\newcommand{\SpecialStringTok}[1]{\textcolor[rgb]{0.31,0.60,0.02}{#1}}
\newcommand{\StringTok}[1]{\textcolor[rgb]{0.31,0.60,0.02}{#1}}
\newcommand{\VariableTok}[1]{\textcolor[rgb]{0.00,0.00,0.00}{#1}}
\newcommand{\VerbatimStringTok}[1]{\textcolor[rgb]{0.31,0.60,0.02}{#1}}
\newcommand{\WarningTok}[1]{\textcolor[rgb]{0.56,0.35,0.01}{\textbf{\textit{#1}}}}
\usepackage{graphicx}
\makeatletter
\def\maxwidth{\ifdim\Gin@nat@width>\linewidth\linewidth\else\Gin@nat@width\fi}
\def\maxheight{\ifdim\Gin@nat@height>\textheight\textheight\else\Gin@nat@height\fi}
\makeatother
% Scale images if necessary, so that they will not overflow the page
% margins by default, and it is still possible to overwrite the defaults
% using explicit options in \includegraphics[width, height, ...]{}
\setkeys{Gin}{width=\maxwidth,height=\maxheight,keepaspectratio}
% Set default figure placement to htbp
\makeatletter
\def\fps@figure{htbp}
\makeatother
\setlength{\emergencystretch}{3em} % prevent overfull lines
\providecommand{\tightlist}{%
  \setlength{\itemsep}{0pt}\setlength{\parskip}{0pt}}
\setcounter{secnumdepth}{-\maxdimen} % remove section numbering
\usepackage{booktabs}
\usepackage{longtable}
\usepackage{array}
\usepackage{multirow}
\usepackage{wrapfig}
\usepackage{float}
\usepackage{colortbl}
\usepackage{pdflscape}
\usepackage{tabu}
\usepackage{threeparttable}
\usepackage{threeparttablex}
\usepackage[normalem]{ulem}
\usepackage{makecell}
\usepackage{xcolor}
\ifLuaTeX
  \usepackage{selnolig}  % disable illegal ligatures
\fi

\begin{document}
\maketitle

\begin{Shaded}
\begin{Highlighting}[]
\FunctionTok{library}\NormalTok{(tidyverse)}
\end{Highlighting}
\end{Shaded}

\begin{verbatim}
## -- Attaching packages --------------------------------------- tidyverse 1.3.1 --
\end{verbatim}

\begin{verbatim}
## v ggplot2 3.3.5     v purrr   0.3.4
## v tibble  3.1.6     v dplyr   1.0.7
## v tidyr   1.1.4     v stringr 1.4.0
## v readr   2.1.1     v forcats 0.5.1
\end{verbatim}

\begin{verbatim}
## -- Conflicts ------------------------------------------ tidyverse_conflicts() --
## x dplyr::filter() masks stats::filter()
## x dplyr::lag()    masks stats::lag()
\end{verbatim}

\begin{Shaded}
\begin{Highlighting}[]
\FunctionTok{library}\NormalTok{(kableExtra)}
\end{Highlighting}
\end{Shaded}

\begin{verbatim}
## 
## Attaching package: 'kableExtra'
\end{verbatim}

\begin{verbatim}
## The following object is masked from 'package:dplyr':
## 
##     group_rows
\end{verbatim}

\begin{Shaded}
\begin{Highlighting}[]
\FunctionTok{library}\NormalTok{(datasets)}
\NormalTok{tinytex}\SpecialCharTok{::}\FunctionTok{install\_tinytex}\NormalTok{()}
\end{Highlighting}
\end{Shaded}

\hypertarget{r-markdown}{%
\subsection{R Markdown}\label{r-markdown}}

\hypertarget{q14}{%
\section{Q14}\label{q14}}

a.) My Data set

\begin{Shaded}
\begin{Highlighting}[]
\NormalTok{data\_set }\OtherTok{\textless{}{-}} \FunctionTok{read.csv}\NormalTok{(}\StringTok{"C:/Users/rache/Documents/STAT 3010/Hw1\_Q14\_data.csv"}\NormalTok{, }\AttributeTok{header =}\NormalTok{ T)}
\end{Highlighting}
\end{Shaded}

Stem Plot

\begin{Shaded}
\begin{Highlighting}[]
\FunctionTok{stem}\NormalTok{(data\_set}\SpecialCharTok{$}\NormalTok{shower\_flow\_rate)}
\end{Highlighting}
\end{Shaded}

\begin{verbatim}
## 
##   The decimal point is at the |
## 
##    2 | 23
##    3 | 2344567789
##    4 | 01356889
##    5 | 00001114455666789
##    6 | 0000122223344456667789999
##    7 | 00012233455555668
##    8 | 02233448
##    9 | 012233335666788
##   10 | 2344455688
##   11 | 2335999
##   12 | 37
##   13 | 8
##   14 | 36
##   15 | 0035
##   16 | 
##   17 | 
##   18 | 9
\end{verbatim}

capture.output(stem(data\_set\$score)) file \textless-
``C:/Users/rache/Documents/STAT 3010/Ex.1.2\_Q14\_data.txt''

b.) Typical flow rate is the flow rate that appears the most. So we
would take one of the values from 6. ex = 6.7

c.) Highly concentrated, only one outlier.

d.) distribution is left skewed since there are more values below the
typical flow rate.

e.) Outlier would be 18.9.

\hypertarget{section}{%
\section{22.)}\label{section}}

a.) sample size = 90 + 190 + 180 + 160 + 120 + 80 + 60 + 40 + 30 + 20 =
970

runners = 10

p = (\# of runners) / (sample size)

p = 10 / 970 = 0.01

\hypertarget{section-1}{%
\section{24.)}\label{section-1}}

My data set

\begin{Shaded}
\begin{Highlighting}[]
\FunctionTok{setwd}\NormalTok{(}\StringTok{"C:/Users/rache/Documents/STAT 3010"}\NormalTok{)}
\NormalTok{data\_set2 }\OtherTok{\textless{}{-}} \FunctionTok{read.csv}\NormalTok{(}\StringTok{"Hw1\_Q24\_data.csv"}\NormalTok{)}

\FunctionTok{head}\NormalTok{(data\_set2)}
\end{Highlighting}
\end{Shaded}

\begin{verbatim}
##   shear_strength
## 1           5434
## 2           5112
## 3           4820
## 4           5378
## 5           5027
## 6           4848
\end{verbatim}

\begin{Shaded}
\begin{Highlighting}[]
\NormalTok{we }\OtherTok{=}\NormalTok{ data\_set2[,}\DecValTok{1}\NormalTok{]}
\end{Highlighting}
\end{Shaded}

My Histogram

\begin{Shaded}
\begin{Highlighting}[]
\FunctionTok{hist}\NormalTok{(we)}
\end{Highlighting}
\end{Shaded}

\includegraphics{Homework-One_files/figure-latex/unnamed-chunk-5-1.pdf}

\hypertarget{section-2}{%
\section{34.)}\label{section-2}}

\begin{Shaded}
\begin{Highlighting}[]
\FunctionTok{setwd}\NormalTok{(}\StringTok{"C:/Users/rache/Documents/STAT 3010"}\NormalTok{)}
\NormalTok{data\_set3 }\OtherTok{\textless{}{-}} \FunctionTok{read.csv}\NormalTok{(}\StringTok{"Hw1\_Q34\_data.csv"}\NormalTok{)}
\end{Highlighting}
\end{Shaded}

a.) Sample mean homes = (6 + 5 + 11 + 33 + 4 + 5 + 80 + 18 + 35 + 17 +
23)/11= 21.55EU/mg

b.) farm homes = (2+15+12+8+8+7+6+19+3+9.8+22+9.6+2+2+0.5)/15 = 8.39

c.) ascending order 4,5,5,6,11,17,18,23,33,35,80 n = 11 ((11+1)/2)th
position = median is 17 EU/mg for urban homes.

ascending order 0.5,2,2,2,3,6,7,8,8,9.6,12,15,19,22 n=15 ((15+1)/2)th
position = median is 8 EU/mg for farm houses.

c.) trimmed mean = (6+5+11+33+5+18+35+17+23)/9= 17.00 for urban houses

trimmed mean = (2+15+12+8+8+7+6+19+3+9.8+9.6+2+2)/13 = 7.95 for farm
houses.

(1 * 100)/11= 9.09\% \textless-urban home trimming percentage (1 *
100)/15 = 6.67\% \textless- farmhouse trimming percentage

\hypertarget{section-3}{%
\section{35.)}\label{section-3}}

\begin{Shaded}
\begin{Highlighting}[]
\NormalTok{data\_set4 }\OtherTok{\textless{}{-}} \FunctionTok{read.csv}\NormalTok{(}\StringTok{"C:/Users/rache/Documents/STAT 3010/Hw1\_Q35\_data.csv.xls"}\NormalTok{)}

\CommentTok{\# a.)}

\NormalTok{x }\OtherTok{\textless{}{-}} \FunctionTok{c}\NormalTok{(data\_set4)}
\NormalTok{result.mean }\OtherTok{\textless{}{-}} \FunctionTok{mean}\NormalTok{(data\_set4,}\AttributeTok{trim =} \DecValTok{0}\NormalTok{)}
\end{Highlighting}
\end{Shaded}

\begin{verbatim}
## Warning in mean.default(data_set4, trim = 0): argument is not numeric or
## logical: returning NA
\end{verbatim}

\begin{Shaded}
\begin{Highlighting}[]
\FunctionTok{print}\NormalTok{(result.mean)}
\end{Highlighting}
\end{Shaded}

\begin{verbatim}
## [1] NA
\end{verbatim}

\hypertarget{section-4}{%
\section{53}\label{section-4}}

a.) n is odd so median is 2.74

Upper fourth meadian is 3.88

3.88 - 2.74 = 1.14

b.) my data set

\begin{Shaded}
\begin{Highlighting}[]
\NormalTok{data\_set53 }\OtherTok{\textless{}{-}} \FunctionTok{read.csv}\NormalTok{(}\StringTok{"C:/Users/rache/Documents/STAT 3010/Hw1\_Q53\_data.csv"}\NormalTok{)}
\end{Highlighting}
\end{Shaded}

boxplot

\begin{Shaded}
\begin{Highlighting}[]
\FunctionTok{boxplot}\NormalTok{(data\_set53, }\AttributeTok{xlab=} \StringTok{"Percentage of Assets"}\NormalTok{, }\AttributeTok{ylab =} \StringTok{"Growth and Blended Funds"}\NormalTok{, }\AttributeTok{ylim=} \FunctionTok{c}\NormalTok{(}\DecValTok{0}\NormalTok{,}\DecValTok{4}\NormalTok{))}
\end{Highlighting}
\end{Shaded}

\includegraphics{Homework-One_files/figure-latex/unnamed-chunk-9-1.pdf}

\#56 data set:

\begin{Shaded}
\begin{Highlighting}[]
\NormalTok{data\_set56 }\OtherTok{\textless{}{-}} \FunctionTok{read.csv}\NormalTok{(}\StringTok{"C:/Users/rache/Documents/STAT 3010/Hw1\_Q56\_data.csv"}\NormalTok{)}
\end{Highlighting}
\end{Shaded}

boxplot

\begin{Shaded}
\begin{Highlighting}[]
\FunctionTok{boxplot}\NormalTok{(data\_set56, }\AttributeTok{xlab=} \StringTok{""}\NormalTok{, }\AttributeTok{ylab =} \StringTok{""}\NormalTok{, }\AttributeTok{ylim=} \FunctionTok{c}\NormalTok{(}\DecValTok{0}\NormalTok{,}\DecValTok{25}\NormalTok{))}
\end{Highlighting}
\end{Shaded}

\includegraphics{Homework-One_files/figure-latex/unnamed-chunk-11-1.pdf}

\hypertarget{section-5}{%
\section{58}\label{section-5}}

a.)

Machine 1 has a small data variation but machine 2 has a high data
variation. Machine 1 also has an outlier but machine 2 does not.

\end{document}
